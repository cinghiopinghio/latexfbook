\documentclass[article(a4paper)]{codedoc}

\usepackage[dvipsnames]{xcolor}
\usepackage[bookmarks,
            bookmarksopen,bookmarksdepth=2
            urlcolor=Green,
            colorlinks=true]{hyperref}
\usepackage[]{url}
%% Fonts
\usepackage[T1]{fontenc}
\usepackage{mathpazo}
\linespread{1.05}
\usepackage{microtype}
\usepackage[]{verbatim}

\usepackage[Q=yes]{examplep}

\def\cmdcolor{red}
\newcommand{\pbitem}[1]{\item[\color{\cmdcolor}\Q{#1}]}
\newcommand{\pbcmd}[1]{\color{\cmdcolor}\Q{#1}}

\title{PHOTOBOOK class}
\author{Mauro}
\date{\today}

\begin{document}
\CodeEscape?
\ProduceFile{photobook.cls}[PhotoBook][v0.1][2015/06/14]

\maketitle

This is a small class intended to help photographers or photo-amateurs to
produce a photobook to eventually print.
It should work with \url{blurb.com}

\tableofcontents

\section{Intro}
\label{sec:intro}

\begin{invisible}
\ProvidesClass{photobook}[?FileDate ?FileVersion of ?FileName]
\NeedsTeXFormat{LaTeX2e}

\end{invisible}

This class provide some useful tricks to easily produce a eyecandy photobook.
It's intended to work with \url{blurb.com} for eventual hardcopy productions.

\begin{invisible}
%% New lenght provided by the class
\newlength{\pbookwidth}
\newlength{\pbookheight}
\newlength{\pbookbleeds}
\setlength{\pbookbleeds}{0.125in}

\end{invisible}

New lengths:
\begin{description}
  \pbitem{\\pbookwidth} Page Width (bleeds included)
  \pbitem{\\pbookheight} Page Height (bleeds included)
  \pbitem{\\pbookbleeds} Page Bleeds
\end{description}


\begin{invisible}
\RequirePackage[cmyk]{xcolor}
\RequirePackage{everypage}

%%%%%%%%%%%%%%%%%%%%%%%%%%%%%%%%%%%
%% Option declaration
%%%%%%%%%%%%%%%%%%%%%%%%%%%%%%%%%%%
\end{invisible}

\section{Class Options}

This class originate from \verb+scrbook+ class, all unlisted options will be
passed to it.

\begin{description}
    \pbitem{magazinepro}: load \url{blurb.com} magazine--pro page sizes.
\begin{invisible}
\DeclareOption{magazinepro}{%
  %% 8.625 x 11.24 in
  %% 621 x 810 pts
  \setlength{\pbookwidth}{8.625in}
  \setlength{\pbookheight}{11.25in}
  %% PDF/X-3:2002 info
  %% The \pageattr values have to be in pt, compare with the page size settings above.
  \pdfpageattr{%
    /MediaBox [0 0 621.00000 810.00000]
    /BleedBox [0.00000 0.00000 621.00000 810.00000]
    /CropBox [0 0 621.00000 810.00000]
    %/TrimBox [0.00000 0.00000 621.00000 810.00000]
    /TrimBox [0.00000 9.00000 612.00000 801.00000]
  }
}
  
\end{invisible}
  \pbitem{showbleeds}: Show bleeds with colored lines
  \begin{invisible}
\DeclareOption{showbleeds}{%
  \AddEverypageHook{\drawbleeds}
}
    
  \end{invisible}
\end{description}

\begin{invisible}
%%All remaining options for the scrbook
\DeclareOption*{%
  \PassOptionsToClass{\CurrentOption}{scrbook}
}
  
%% Execute default options
\ExecuteOptions{magazinepro}
%% Process all other options
\ProcessOptions\relax
\LoadClass{scrbook}

%%%%%%%%%%%%%%%%%%%%%%%%%%%%%%%%%%%
%% Load packages and latex code
%%%%%%%%%%%%%%%%%%%%%%%%%%%%%%%%%%%
\RequirePackage{geometry}
\geometry{%
  paperwidth=\pbookwidth,
  paperheight=\pbookheight,
            % amount we 'lose' (visually) due to the binding
            % left and right pages are offset a bit to the sides
            % (typically looks better for most kinds of bindings)
            bindingoffset=0.25in,
            % set total to the dimensions of the printed part
            %total={8in,7.75in},
            % include header/footer/margin notes in printed area
            twoside, includeall, nomarginpar,
            ignorehead=false, ignorefoot=false, ignoremp=false,
            % center printed area on page
            %vcentering,hcentering,
            top = 1in, bottom=1in,
            inner = 1.0in, outer=1.7in,
}

\RequirePackage{titling}
%%%%%%%%%%%%%%%%%%%%%%%%%%%%%%%%%%%
%% PDF/X-3:2002 info
%%%%%%%%%%%%%%%%%%%%%%%%%%%%%%%%%%%
\pdfobjcompresslevel=0%
\pdfminorversion=3%
\pdfcatalog{%
  /PageMode /UseNone
  /OutputIntents [ <<
    /Info (none)
    /Type /OutputIntent
    /S /GTS_PDFX
    /OutputConditionIdentifier (Blurb.com)
    /RegistryName (http://www.color.org/)
  >> ]
}%
\AtBeginDocument{%
  %% wait for title and author to be set by user
  %% Replace the title, author, etc. information accordingly to your book.
  \pdfinfo{%
    /Title (\thetitle)
    /Author (\theauthor)
    /Subject (photography)
    /Keywords (photography)
    /GTS_PDFXVersion (PDF/X-3:2002)
  }%
}

\RequirePackage{tikz}
\RequirePackage{tikzpagenodes}
% provide nodes as:
% - current page text area
% - current page marginpar area
% - current page header area
% - current page footer area
\usetikzlibrary{calc, positioning, matrix}
\RequirePackage{xkeyval}

\end{invisible}

\section{Commands}
\label{sec:commands}

\begin{description}
    \pbitem{\\putbox[]\{\}}
    \pbitem{\\putmatrix[]\{\}}
    \pbitem{\\caption[]\{\}}
\end{description}

\begin{invisible}
%%%%%%%%%%%%%%%%%%%%%%%%%%%%%%%%%%%
%% Provided commands
%%%%%%%%%%%%%%%%%%%%%%%%%%%%%%%%%%%

\makeatletter

\define@cmdkeys{putbox}{anchor,pageanchor}[center]
\define@cmdkeys{putbox}{xshift,yshift}[0pt]
\define@cmdkeys{putbox}{xsep,ysep}[.1in]
\define@cmdkeys{putbox}{pagenode}[current page]
\presetkeys{putbox}{anchor,pageanchor,pagenode,
  xshift,yshift,xsep,ysep}{}
\def\putbox{\@ifnextchar[{\@putbox}{\@putbox[]}}
\def\@putbox[#1]#2{%
  \setkeys{putbox}{#1}
  \begin{tikzpicture}[overlay,
                      remember picture,
                      shift=(\cmdKV@putbox@pagenode.\cmdKV@putbox@pageanchor)]
    \node[anchor=\cmdKV@putbox@anchor,inner sep=0pt] 
      at (\cmdKV@putbox@xshift,\cmdKV@putbox@yshift) (lastputbox) {#2};
  \end{tikzpicture}
}

\def\putmatrix{\@ifnextchar[{\@putmatrix}{\@putmatrix[]}}
\def\@putmatrix[#1]#2{%
  \setkeys{putbox}{#1}
  \begin{tikzpicture}[overlay,
                      remember picture,
                         every node/.style={inner sep=0pt},
                      shift=(\cmdKV@putbox@pagenode.\cmdKV@putbox@pageanchor)]
    \matrix [anchor=\cmdKV@putbox@anchor,
             ampersand replacement=\&,
             matrix of nodes,
             column sep=\cmdKV@putbox@xsep,
             row sep=\cmdKV@putbox@ysep,
             ] 
        at (\cmdKV@putbox@xshift,\cmdKV@putbox@yshift)  (lastputbox) {#2};
  \end{tikzpicture}
}

\renewcommand\caption[1]{%
  \putbox[anchor=north,
          pagenode=lastputbox,
          pageanchor=south,
          yshift=-.2in,
          ] {%
            \begin{minipage}{\textwidth}
              #1
            \end{minipage}
          }
}

\newcommand\drawbleeds{
  \begin{tikzpicture}[overlay,
                      remember picture,
                      shift=(current page.south west)]

    \ifodd\c@page
    \draw[black!50] (0,\pbookbleeds) rectangle
    ($(\pbookwidth-\pbookbleeds,\pbookheight-\pbookbleeds)$); 
    \else
      \draw[black!50] (\pbookbleeds,\pbookbleeds) rectangle
                   ($(\pbookwidth,\pbookheight-\pbookbleeds)$); 
    \fi
  \end{tikzpicture}
}

\makeatother
\endinput

\end{invisible}
\end{document}
